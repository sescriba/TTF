% Chapter 1

\chapter{Introducción general} % Main chapter title

\label{Chapter1} % For referencing the chapter elsewhere, use \ref{Chapter1} 
\label{IntroGeneral}

En este capítulo se introducen conceptos básicos de los sistemas de iluminación modernos, también se presentan los objetivos, la motivación, los alcances de este trabajo y se analiza el estado del arte.

%----------------------------------------------------------------------------------------

% Define some commands to keep the formatting separated from the content 
\newcommand{\keyword}[1]{\textbf{#1}}
\newcommand{\tabhead}[1]{\textbf{#1}}
\newcommand{\code}[1]{\texttt{#1}}
\newcommand{\file}[1]{\texttt{\bfseries#1}}
\newcommand{\option}[1]{\texttt{\itshape#1}}
\newcommand{\grados}{$^{\circ}$}

%----------------------------------------------------------------------------------------

%\section{Introducción}

%----------------------------------------------------------------------------------------
\section{Introducción a los sistemas de iluminación}

A partir de la llegada de los LEDs ({\it{Light Emissor Doide}}) de alta potencia al mercado de la electrónica, los sistemas de iluminación del hogar (y en general) experimentaron una verdadera revolución comercial y tecnológica. Los bajos costos y la posibilidad de control generaron un sin fin innovador que sigue vigente hoy en día. En una casa tipo es común observar lámparas LED de colores brillantes capaces de ser controladas con un teléfono celular y producir un efecto visual poco pensado algunas décadas en el pasado o, también, su capacidad de ser encendidas o apagadas con un simple ''click programado''. Sin embargo, aún constituye un desafío para los ingenieros integrar estos sistemas de iluminación a la arquitectura edilicia manteniendo eficiencia, estética y bajo costo.

\subsection{Iluminación en escaleras modernas}

Uno de los mayores problemas de la iluminación de los hogares modernos es lograr un efecto lumínico vistoso sobre espacios asimétricos, con un número elevado de bordes irregulares donde las sombras juegan un desafío. Un claro ejemplo son las escaleras, donde los propios escalones proyectan sombras sobre sus predecesores o antecesores. 

La ingeniería en iluminación se encarga de determinar la cantidad de luz que requiere un espacio, en función de sus características y de las normas vigentes. El Código Residencial Internacional \citep{CODRED} recomienda 100 lúmenes por pie cuadrado (alrededor de 100 centímetros cuadrados) en sectores de interior con escaleras y cuya temperatura de color ronde entre 2700 y 3000 kelvin.

Sin embargo, hay otro aspecto que este tipo de códigos no contemplan del todo: la eficiencia energética. Este factor es fundamental a la hora de efectuar una obra lumínica en un hogar, ya que define la estrategia constructiva de la escalera.

\subsection{Eficiencia energética}

Tanto la disponibilidad de energía eléctrica como la potencia requerida por el sistema de iluminación constituyen aspectos clave en el diseño de una instalación lumínica para una escalera. Si se dispone de pocos puntos de alimentación eléctrica, la oportunidad de agregar puntos de luz al espacio estará completamente limitado, lo que afecta la capacidad de cubrirlo con los niveles de lúmenes adecuados. Además, deben considerarse el consumo total sobre dicho acceso energético y la capacidad de la infraestructura para soportarlo.

A pesar de estos inconvenientes, los avances tecnológicos han generado soluciones a partir de la invención de las tiras LED, cuyo consumo energético es alrededor de un 90\% menor que el de un sistema de iluminación tradicional. De manera general, estas tiras logran iluminar 100 lúmenes por vatio, mientras que una bombilla incandescente ofrece aproximadamente 20 lúmenes por vatio. Además, su extensión lineal permite distribuir la energía eléctrica desde un único punto de alimentación a lo largo de toda la escalera sin requerir cableado adicional.

%----------------------------------------------------------------------------------------

\section{Estado del arte}

En esta sección se presentan algunos productos similares al desarrollado en este documento que se encuentran disponibles en el mercado. Se realiza un breve análisis de las prestaciones y costos, y se describen las diferencias y similitudes con la solución propuesta en este trabajo.
 
\subsection{SuperlightingLED}

La empresa de tecnología LED SuperlightingLED \citep{SuperlightingLED} ofrece en el mercado internacional un sistema de iluminación de escaleras con tecnología de tiras LEDs y sensores de movimientos (SCSSLKIT-12V-COB). 

Según la información publicada en su página de ventas, el sistema incluye un controlador ES32, que actúa como unidad de control principal, y dos sensores de movimiento ubicados en los extremos de la escalera. Es importante aclarar que el ES32 no es un microcontrolador, sino un controlador dedicado para tiras LED, diseñado para gestionar efectos de iluminación, la activación por sensores y la intensidad del brillo. Además, cuenta con una fuente de alimentación de 12 V y las tiras LED (de color y temperatura variables) para cubrir hasta 32 escalones de un ancho total de 1,2 m cada uno y que deben cablearse hasta la unidad controladora. 

El sistema se comercializa en dos versiones principales: una para interior (IP20) y otra para exterior (IP67). Adicionalmente, puede incluir perfiles de aluminio para facilitar su instalación. 

En las figuras \ref{fig:spled1} y \ref{fig:spled2} se pueden apreciar los componentes de este sistema que actualmente tiene un costo de unos USD 160 en su versión más económica y hasta USD 630 en su versión premium.

\begin{figure}[H]
	\centering
	\includegraphics[width=7cm, height=7cm]{./Figures/superlightingLED.png}
	\caption{Sistema de SuperlightingLED y su conexión.\protect\footnotemark}

	\label{fig:spled1}
\end{figure}
\footnotetext{\url{https://www.superlightingled.com/}}


\begin{figure}[H]
    \centering
    \includegraphics[width=7cm, height=7cm]{./Figures/superlightingLED2.png}
    \caption{Componentes incluidos en el paquete básico.\protect\footnotemark}
    \label{fig:spled2}
\end{figure}
\footnotetext{\url{https://www.superlightingled.com/}}

\subsection{GIDEALED}

Otro producto presente en el mercado que se lo puede vincular con el trabajo desarrollado en esta memoria, es el sistema de iluminación de escaleras de la empresa GIDEALED \citep{GIDEALED} modelo RL-STEP-09. Este sistema es apto para hasta 20 escalones de un metro de ancho, funciona a 12 V y consta de dos sensores de movimientos del tipo PIR (\textit{Passive Infrared Sensor}, por sus siglas en inglés) a instalar en los extremos de la escalera, además de un control remoto infrarrojo para el encendido y apagado. Las tiras LED incluidas en el kit son de color fijo, de temperatura de color de 3000 K, e incorporan control de brillo e intensidad. No cuentan con protección frente al agua o humedad. La figura \ref{fig:gidealed} muestra los componentes del kit, que se encuentra a la venta en la República Argentina por un precio aproximado de \$210000.

\begin{figure}[H]
	\centering
	\includegraphics[width=7cm, height=7cm]{./Figures/GIDEALED.png}
	\caption{Sistema para escaleras de GIDEALED.\protect\footnotemark}
	\label{fig:gidealed}
\end{figure}
\footnotetext{\url{https://www.amazon.com/stores/GIDEALED/page/55B6DBC4-21FB-439B-AC80-13FDE4121084}}

\subsection{Análisis del estado del arte y aportes del desarrollo}

Luego de realizar un análisis exhaustivo de los distintos productos disponibles en el mercado, se concluye que, si bien existen soluciones que abordan la problemática planteada, la mayoría de ellas carece del grado de innovación y personalización que se busca en este desarrollo. La incorporación de sensores individuales en cada escalón permite un control de iluminación más preciso y adaptativo, constituyendo uno de los principales factores diferenciales del sistema propuesto.
A ello se suman su bajo costo de implementación, la posibilidad de instalación directa por parte del usuario sin asistencia técnica y la integración de tecnologías modernas de control.

En la siguiente lista se destacan las principales ventajas del sistema frente a los productos analizados:
\begin{itemize}
    \item Bajo costo de desarrollo y mantenimiento.
    \item Instalación sencilla, sin necesidad de herramientas especializadas.
    \item Mayor precisión y personalización del control lumínico.
    \item Posibilidad de control mediante teléfono celular vía conexión Bluetooth.
    \item Configuración en conexión serie de los módulos de luz.
    \item Compatibilidad con escaleras de hasta 64 escalones.
\end{itemize}
%----------------------------------------------------------------------------------------

\section{Motivación}

Si bien existen múltiples sistemas de iluminación destinados al hogar, las escaleras resultan un ámbito poco explorado en cuanto a propuestas innovadoras de diseño lumínico. Este espacio, por sus características arquitectónicas y su uso cotidiano, ofrece un amplio potencial para la experimentación con nuevas formas de iluminación funcional y estética. En este sentido, los avances en electrónica y domótica desempeñan un papel fundamental al permitir el desarrollo de sistemas más inteligentes, eficientes y adaptables. Gracias a ellos, el usuario no solo puede decidir qué iluminar, sino también cómo hacerlo, ajustando la intensidad, el color o el comportamiento de la luz según sus preferencias o necesidades específicas.

Por otra parte, la disponibilidad de sensores, actuadores y microcontroladores de bajo costo ha permitido el acceso a tecnologías que antes eran exclusivas de instalaciones complejas o costosas. A esto se suman los sistemas de comunicación inalámbrica, que facilitan la implementación sin requerir cableados extensos ni modificaciones estructurales. Estos factores en conjunto constituyen una motivación central para generar un nuevo módulo lumínico transformable que responda a las necesidades de un usuario particular. 


%----------------------------------------------------------------------------------------

\section{Objetivos y alcances}

Este trabajo de desarrollo pretende integrar los avances e innovaciones tecnológicas en la iluminación del hogar agregando valor a bajo costo y control por parte del usuario. Es así que, integrando los conocimientos adquiridos a lo largo de la Carrera de Especialización en Sistemas Embebidos de la FIUBA, se desarrolla el firmware en lenguaje de programación C de un sistema embebido basado en tiras LED y controlado mediante sensores de proximidad que permita iluminar una escalera de una cantidad fija de escalones. También es parte de este trabajo la incorporación de un control mediante tecnología Bluetooth de la tonalidad de la luz, brillo y color, así como su encendido o apagado de acuerdo a la circulación del usuario por la escalera.

Por otro lado, este trabajo no contempla el desarrollo de un hardware integral, sino la elaboración de un prototipo funcional basado en componentes provistos por fabricantes comerciales. Tampoco se contempla el desarrollo de una interfaz de usuario ni soporte postventa.
%----------------------------------------------------------------------------------------