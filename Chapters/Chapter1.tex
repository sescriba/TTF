% Chapter 1

\chapter{Introducción general} % Main chapter title

\label{Chapter1} % For referencing the chapter elsewhere, use \ref{Chapter1} 
\label{IntroGeneral}

En este capítulo se introduce conceptos básicos de los sistemas de iluminación modernos y se presentan los objetivos y alcances de este proyecto.

%----------------------------------------------------------------------------------------

% Define some commands to keep the formatting separated from the content 
\newcommand{\keyword}[1]{\textbf{#1}}
\newcommand{\tabhead}[1]{\textbf{#1}}
\newcommand{\code}[1]{\texttt{#1}}
\newcommand{\file}[1]{\texttt{\bfseries#1}}
\newcommand{\option}[1]{\texttt{\itshape#1}}
\newcommand{\grados}{$^{\circ}$}

%----------------------------------------------------------------------------------------

%\section{Introducción}

%----------------------------------------------------------------------------------------
\section{Introducción a los sistemas de iluminación}

A partir de la llegada de los LEDs ({\it{Light Emissor Doide}}) de alta potencia al mercado de la electrónica, los sistemas de iluminación del hogar (y en general) experimentaron una verdadera revolución comercial y tecnológica. Los bajos costos y la posibilidad de control de los mismos generaron un sin fin innovador que sigue vigente hoy en día. En una casa tipo es común observar lámparas LED de colores brillantes capaces de ser controladas con un teléfono celular produciendo un efecto visual poco pensado algunas décadas en el pasado o, también, su capacidad de ser encendidas o apagadas con un simple "click programado". Sin embargo es aún un desafío para los ingenieros poder integrar estos sistemas de iluminación a la arquitectura edilicia y que los mismos sean eficientes, vistosos y de bajo costo.

\subsection{Iluminando una escalera moderna}

Uno de los mayores problemas de la iluminación de los hogares modernos es lograr un efecto lumínico vistoso sobre espacios asimétricos, con un número elevado de bordes irregulares donde las sombras juegan un desafío. Un claro ejemplo son las escaleras, donde los propios escalones proyectan sombras sobre sus predecesores o antecesores. La ingeniería en iluminación se encargar de calcular para estos casos la cantidad de lúmenes (es decir, la cantidad de luz producida por una fuente) que debe tener un cierto volumen del espacio considerando sus deformaciones y en sintonía con normas específicas. El Código Residencial Internacional recomienda 100 lúmenes por pie cuadrado (alrededor de 100 centímetros cuadrados) en sectores de interior con escaleras y cuya temperatura de color ronde entre 2700 y 3000 Kelvin.

Sin embargo, hay otro aspecto que estos códigos no contemplan del todo: la eficiencia energética. Este factor es fundamental a la hora de efectuar una obra lumínica en un hogar, ya que definirá la estrategia constructiva de la escalera.

\subsection{Eficiencia energética}

Tanto el acceso a la energía eléctrica como la cantidad de watts que se consuman en un sistema de iluminación van a generar consideraciones importantes a la hora de pensar un sistema lumínico para una escalera. Si se dispone de pocos accesos a la energía, la oportunidad de agregar puntos de luz al espacio estará completamente limitado, afectado la capacidad de cubrir el mismo con los niveles de lúmenes adecuados. Además, deben considerarse el consumo total sobre dicho acceso energético y que la infraestructura lo soporte. 

A pesar de estos inconvenientes, los avances tecnológicos han generado soluciones a partir de la invención de las tiras LED, cuyo consumo energético es alrededor de un 90\% menor a un sistema de iluminación tradicional. A grandes rasgos, estas tiras logran iluminar 100 lúmenes por vatio de energía, mientras que una bombilla incandescente ofrecen 20 lúmenes por vatio. Por otro lado, su desarrollo longitudinal permite transportar la electricidad desde un único punto de acceso energético hasta el fin del espacio a iluminar sin el adicional de cableado.

%----------------------------------------------------------------------------------------

\section{Estado del arte}

En esta sección se presentan algunos productos similares al aquí desarrollado disponibles en el mercado. Se realiza un pequeño análisis de las prestaciones y costos y se exponen las diferencias y similitudes con el propuesto en esta memoria.
 
\subsection{SuperlightingLED}

La empresa de tecnología LED SuperlightingLED ofrece en el mercado internacional un sistema de iluminación de escaleras con tecnología de tiras LEDs y sensores de movimientos (SCSSLKIT-12V-COB). De acuerdo a su pagina de ventas este producto consta de un microcontrolador ES32 y dos sensores de movimiento ubicados al extremo de la escalera donde se instala. Además cuenta con una fuente de alimentación de 12V y las tiras LED (de color y temperatura variables) para cubrir hasta 32 escalones de un ancho total de 1,2 metros cada uno las cuales debe estar cableadas a la central de controladora. Cuenta con dos versiones principales tanto para interior (IP20) como para exterior (IP67) y también puede o no incluir perfiles de aluminio para facilitar su instalación. En las figuras \ref{fig:spled1} y \ref{fig:spled2} se puede apreciar los componentes de este sistema que al día de la fecha tiene un costo de unos U\$D160 en su versión más económica y hasta U\$D630 en su versión premium. Para más información puede visitar la página web de \textit{SuperlightingLED}\footnote{\url{https://www.superlightingled.com}}.
\begin{figure}[H]
	\centering
	\includegraphics[width=10cm, height=10cm]{./Figures/superlightingLED.png}
	\caption{Sistema de SuperlightingLED y su conexión.}
	\label{fig:spled1}
\end{figure}

\begin{figure}[H]
	\centering
	\includegraphics[width=10cm, height=10cm]{./Figures/superlightingLED2.png}
	\caption{Componentes incluidos en el paquete básico.}
	\label{fig:spled2}
\end{figure}

\subsection{GIDEALED}

Otro producto presente en el mercado al cual se lo puede vincular con el proyecto tratado en esta memoria es el sistema de iluminación de escaleras de la empresa \textit{GIDEALED}\footnote{\url{https://www.amazon.com/stores/GIDEALED/page/55B6DBC4-21FB-439B-AC80-13FDE4121084}} modelo RL-STEP-09. Este sistema apto para hasta 20 escalones de un metro de ancho, posee un voltaje de funcionamiento de 12V y consta de dos sensores de movimientos del tipo PIR (\textit{Passive Infrared Sensor}, por sus siglas en inglés) a instalar en los extremos de la escalera además de un control remoto infrarrojo para el control del encendido y apagado. Las tiras LED incluidas en el kit son de color fijo de temperatura 3000K con control de brillo e intensidad sin protección al agua o humedad. La figura \ref{fig:gidealed} muestra los componentes del kit que se encuentra a la venta en la República Argentina por el precio de ARS210000.
\begin{figure}[H]
	\centering
	\includegraphics[width=10cm, height=10cm]{./Figures/GIDEALED.png}
	\caption{Sistema para escaleras de GIDEALED.}
	\label{fig:gidealed}
\end{figure}

%----------------------------------------------------------------------------------------

\section{Motivación}

Si bien existen múltiples sistemas de iluminación para el hogar, las escaleras son aún un territorio poco explorado y con gran versatilidad a diseños. Además, los avances de la electrónica juegan un rol fundamental para que estos diseños sea innovadores, con capacidad de control sobre el qué y el cómo se desea iluminar por parte del usuario.

La amplia gama de sensores y actuadores de bajo costo, así como los sistemas de comunicación sin cableado, son la motivación para generar un nuevo módulo lumínico transformable que responda a las necesidades de un usuario particular. 


%----------------------------------------------------------------------------------------

\section{Objetivos y alcances}

Este trabajo de desarrollo pretende integrar los avances e innovaciones tecnológicas en la iluminación del hogar agregando valor a bajo costo y control por parte del usuario. Es así que, integrando los conocimientos adquiridos a lo largo de la Carrera de Especialidad en Sistemas Embebidos de la Universidad de Buenos Aires, se desarrolla el Firmware en lenguaje de programación C de un sistema embebido basado en tiras LED y controlado mediante sensores de proximidad que permita iluminar una escalera de una cantidad fija de escalones. También es parte de este proyecto la incorporación de un control mediante tecnología Bluetooth de la tonalidad de la luz, brillo y color, así como su encendido o apagado de acuerdo a la circulación del usuario por la escalera.

Por otro lado, no es parte de este trabajo el desarrollo de un Hardware integral, si no que se produce un prototipo funcional utilizando componentes de fabricantes comerciales. Tampoco es parte de este proyecto el desarrollo del código de una interfaz de usuario ni brindar soporte a la misma.

%----------------------------------------------------------------------------------------