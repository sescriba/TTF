% Chapter 1

\chapter{Introducción general} % Main chapter title

\label{Chapter1} % For referencing the chapter elsewhere, use \ref{Chapter1} 
\label{IntroGeneral}

En este capítulo se introduce conceptos básicos de los sistemas de iluminación modernos, y se presentan los objetivos y alcances de este trabajo.

%----------------------------------------------------------------------------------------

% Define some commands to keep the formatting separated from the content 
\newcommand{\keyword}[1]{\textbf{#1}}
\newcommand{\tabhead}[1]{\textbf{#1}}
\newcommand{\code}[1]{\texttt{#1}}
\newcommand{\file}[1]{\texttt{\bfseries#1}}
\newcommand{\option}[1]{\texttt{\itshape#1}}
\newcommand{\grados}{$^{\circ}$}

%----------------------------------------------------------------------------------------

%\section{Introducción}

%----------------------------------------------------------------------------------------
\section{Introducción a los sistemas de iluminación}

A partir de la llegada de los LEDs ({\it{Light Emissor Doide}}) de alta potencia al mercado de la electrónica, los sistemas de iluminación del hogar (y en general) experimentaron una verdadera revolución comercial y tecnológica. Los bajos costos y la posibilidad de control generaron un sin fin innovador que sigue vigente hoy en día. En una casa tipo es común observar lámparas LED de colores brillantes capaces de ser controladas con un teléfono celular produciendo un efecto visual poco pensado algunas décadas en el pasado o, también, su capacidad de ser encendidas o apagadas con un simple "click programado". Sin embargo, aún constituye un desafío para los ingenieros integrar estos sistemas de iluminación a la arquitectura edilicia manteniendo eficiencia, estética y bajo costo.

\subsection{Iluminando una escalera moderna}

Uno de los mayores problemas de la iluminación de los hogares modernos es lograr un efecto lumínico vistoso sobre espacios asimétricos, con un número elevado de bordes irregulares donde las sombras juegan un desafío. Un claro ejemplo son las escaleras, donde los propios escalones proyectan sombras sobre sus predecesores o antecesores. La ingeniería en iluminación se encarga de determinar la cantidad de luz que requiere un espacio, en función de sus características y de las normas vigentes. El Código Residencial Internacional recomienda 100 lúmenes por pie cuadrado (alrededor de 100 centímetros cuadrados) en sectores de interior con escaleras y cuya temperatura de color ronde entre 2700 y 3000 Kelvin.

Sin embargo, hay otro aspecto que estos códigos no contemplan del todo: la eficiencia energética. Este factor es fundamental a la hora de efectuar una obra lumínica en un hogar, ya que define la estrategia constructiva de la escalera.

\subsection{Eficiencia energética}

Tanto la disponibilidad de energía eléctrica como la potencia requerida por el sistema de iluminación constituyen aspectos clave en el diseño de una instalación lumínica para una escalera. Si se dispone de pocos accesos a la energía, la oportunidad de agregar puntos de luz al espacio estará completamente limitado, afectado la capacidad de cubrirlo con los niveles de lúmenes adecuados. Además, deben considerarse el consumo total sobre dicho acceso energético y que la infraestructura lo soporte. 

A pesar de estos inconvenientes, los avances tecnológicos han generado soluciones a partir de la invención de las tiras LED, cuyo consumo energético es alrededor de un 90\% menor a un sistema de iluminación tradicional. De manera general, estas tiras logran iluminar 100 lúmenes por vatio de energía, mientras que una bombilla incandescente ofrecen 20 lúmenes por vatio. Además, su extensión lineal permite distribuir la energía eléctrica desde un único punto de alimentación a lo largo de toda la escalera sin requerir cableado adicional.

%----------------------------------------------------------------------------------------

\section{Estado del arte}

En esta sección se presentan algunos productos similares al aquí desarrollado disponibles en el mercado. Se realiza un pequeño análisis de las prestaciones y costos y se exponen las diferencias y similitudes con el expuesto en esta memoria.
 
\subsection{SuperlightingLED}

La empresa de tecnología LED SuperlightingLED \citep{SuperlightingLED} ofrece en el mercado internacional un sistema de iluminación de escaleras con tecnología de tiras LEDs y sensores de movimientos (SCSSLKIT-12V-COB). De acuerdo a su página de ventas este producto consta de un microcontrolador ES32 y dos sensores de movimiento ubicados al extremo de la escalera donde se instala. Además, cuenta con una fuente de alimentación de 12 V y las tiras LED (de color y temperatura variables) para cubrir hasta 32 escalones de un ancho total de 1,2 m cada uno y que debe estar cableadas a la central de controladora. Cuenta con dos versiones principales tanto para interior (IP20) como para exterior (IP67) y también puede o no incluir perfiles de aluminio para facilitar su instalación. En las figuras \ref{fig:spled1} y \ref{fig:spled2} se puede apreciar los componentes de este sistema que actualmente tiene un costo de unos \$160 en su versión más económica y hasta \$630 en su versión premium. Para más información puede visitar la página web de SuperlightingLED.
\begin{figure}[H]
	\centering
	\includegraphics[width=10cm, height=10cm]{./Figures/superlightingLED.png}
	\caption{Sistema de SuperlightingLED y su conexión.}
	\label{fig:spled1}
\end{figure}

\begin{figure}[H]
	\centering
	\includegraphics[width=10cm, height=10cm]{./Figures/superlightingLED2.png}
	\caption{Componentes incluidos en el paquete básico.}
	\label{fig:spled2}
\end{figure}

\subsection{GIDEALED}

Otro producto presente en el mercado que se lo puede vincular con el proyecto tratado en esta memoria es el sistema de iluminación de escaleras de la empresa GIDEALED \citep{GIDEALED} modelo RL-STEP-09. Este sistema apto para hasta 20 escalones de un metro de ancho, posee un voltaje de funcionamiento de 12 V y consta de dos sensores de movimientos del tipo PIR (\textit{Passive Infrared Sensor}, por sus siglas en inglés) a instalar en los extremos de la escalera además de un control remoto infrarrojo para el control del encendido y apagado. Las tiras LED incluidas en el kit son de color fijo de temperatura 3000 K con control de brillo e intensidad sin protección al agua o humedad. La figura \ref{fig:gidealed} muestra los componentes del kit que se encuentra a la venta en la República Argentina por el precio de \$210000.
\begin{figure}[H]
	\centering
	\includegraphics[width=10cm, height=10cm]{./Figures/GIDEALED.png}
	\caption{Sistema para escaleras de GIDEALED.}
	\label{fig:gidealed}
\end{figure}

\subsection{Análisis y conclusiones}

Luego de realizar un análisis exhaustivo de los distintos productos disponibles en el mercado, se concluye que, si bien existen soluciones que abordan la problemática planteada, la mayoría de ellas carece del grado de innovación y personalización que se busca en este desarrollo. La incorporación de sensores individuales en cada escalón permite un control de iluminación más preciso y adaptativo, constituyendo uno de los principales factores diferenciales del sistema propuesto.
A ello se suman su bajo costo de implementación, la posibilidad de instalación directa por parte del usuario sin asistencia técnica, y la integración de tecnologías modernas de control.

En la siguiente lista se destacan las principales ventajas del sistema frente a los productos analizados:
\begin{itemize}
    \item Bajo costo de desarrollo y mantenimiento.
    \item Instalación sencilla, sin necesidad de herramientas especializadas.
    \item Mayor precisión y personalización del control lumínico.
    \item Posibilidad de control mediante teléfono celular vía conexión Bluetooth.
    \item Configuración en conexión serie de los módulos de luz.
    \item Compatibilidad con escaleras de hasta 64 escalones.
\end{itemize}
%----------------------------------------------------------------------------------------

\section{Motivación}

Si bien existen múltiples sistemas de iluminación destinados al hogar, las escaleras continúan siendo un ámbito poco explorado en cuanto a propuestas innovadoras de diseño lumínico. Este espacio, por sus características arquitectónicas y su uso cotidiano, ofrece un amplio potencial para la experimentación con nuevas formas de iluminación funcional y estética. En este sentido, los avances en electrónica y domótica desempeñan un papel fundamental al permitir el desarrollo de sistemas más inteligentes, eficientes y adaptables. Gracias a ellos, el usuario no solo puede decidir qué iluminar, sino también cómo hacerlo, ajustando la intensidad, el color o el comportamiento de la luz según sus preferencias o necesidades específicas.

Por otra parte, la disponibilidad de sensores, actuadores y microcontroladores de bajo costo ha permitido el acceso a tecnologías que antes eran exclusivas de instalaciones complejas o costosas. A esto se suman los sistemas de comunicación inalámbrica, que facilitan la implementación sin requerir cableados extensos ni modificaciones estructurales. Estos factores en conjunto constituyen una motivación central para generar un nuevo módulo lumínico transformable que responda a las necesidades de un usuario particular. 


%----------------------------------------------------------------------------------------

\section{Objetivos y alcances}

Este trabajo de desarrollo pretende integrar los avances e innovaciones tecnológicas en la iluminación del hogar agregando valor a bajo costo y control por parte del usuario. Es así que, integrando los conocimientos adquiridos a lo largo de la Carrera de Especialidad en Sistemas Embebidos de la Universidad de Buenos Aires, se desarrolla el firmware en lenguaje de programación C de un sistema embebido basado en tiras LED y controlado mediante sensores de proximidad que permita iluminar una escalera de una cantidad fija de escalones. También es parte de este proyecto la incorporación de un control mediante tecnología Bluetooth de la tonalidad de la luz, brillo y color, así como su encendido o apagado de acuerdo a la circulación del usuario por la escalera.

Por otro lado, no es parte de este trabajo el desarrollo de un Hardware integral, si no que se produce un prototipo funcional utilizando componentes de fabricantes comerciales. Tampoco es parte de este proyecto el desarrollo del código de una interfaz de usuario ni brindar soporte a la misma.
%----------------------------------------------------------------------------------------