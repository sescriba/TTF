\chapter{Introducción específica} % Main chapter title

\label{Chapter2}

%----------------------------------------------------------------------------------------
%	SECTION 1
%----------------------------------------------------------------------------------------
En este capitulo se mencionan los componentes, software y programas de terceros utilizados para el desarrollo del sistema de esta memoria.

\section{Componentes del sistema}
\label{sec:ejemplo}

El sistema de iluminación cuenta con un conjunto de componentes necesarios para su funcionamiento, que van desde lenguajes de programación hasta pequeños componentes electrónicos. A continuación se hace mención a cada uno de ellos con una breve descripción. 

\subsection{Hardware}

\subsubsection{Microcontrolador}
Este proyecto utiliza un microcontrolador de 32 bits de la marca STMicrocontroller modelo F4 basado en el núcleo Arm Cortex-M4 con un frecuencia de trabajo de hasta 180 MHz. Incluye una unidad de punto flotante de precisión simple e incorporan memorias embebidas de alta velocidad (memoria Flash de 2 Mbytes y hasta 256 Kbytes de SRAM). Cuenta con un gran numero de entradas/salidas, convertidores analógicos digitales y hasta 12 temporizadores (incluyendo dos para modulación de pulsos). Posee además de un gran numero de interfaces de comunicación entre las que se destacan UART ({\it{Universal Asynchronous Receiver-Transmitter}}), I2C ({\it{Inter-Integrated Circuit}}), SMBus ({\it{System Management Bus}}) y SPI ({\it{Serial Peripheral Interface}}). Para el desarrollo de este proyecto se da uso a la placa de evaluación Núcleo de la marca STM modelo F429ZI que dispone de todos los periféricos necesarios para el uso del microcontrolador y sus componentes.
\begin{figure}[H]
	\centering
	\includegraphics[width=5cm, height=5cm]{./Figures/nucleo.jpg}
	\caption{Placa de evaluación Núcleo de STM.}
	\label{fig:nucleo}
\end{figure}

\subsubsection{Tira LED pixel}
Para el desarrollo del sistema de iluminación se dispone de una tira LED del tipo pixel. La misma consta conjuntos de tres diodos emisores de luz (uno para cada color primario RGB) con un controlador modelo WS2812B conectados en cadena. Este controlador incluye un registro de datos digitales, un circuito de amplificación y reformado de señal, en conjunto con un oscilador interno de alta precisión. Utiliza un protocolo de comunicación NZR ({\it{Non Return Zero}}) de un solo cable, donde el puerto de entrada recibe los datos enviados por el microcontrolador, son almacenados en un registro interno y transferidos por el puerto de salida al siguiente dispositivo conectado en la cadena, como se ilustra en la figura \ref{fig:led}.  El dato consta de 24 pulsos (8 por color), donde un cero lógico debe estar compuesto por 0.4 $\mu$s en alto y 0.85 $\mu$s en bajo y un uno lógico por 0.8 $\mu$s en alto y 0.45 $\mu$s en bajo.

El voltaje de trabajo es de 5 V y el consumo depende de la cantidad de LEDs de la tira, con un consumo cercano a 1 uA por LED.

\begin{figure}[H]
	\centering
	\includegraphics[width=5cm, height=5cm]{./Figures/leds.png}
	\caption{Conexión en cadena de los dispositivos WS2812B.}
	\label{fig:led}
\end{figure}

\subsubsection{Módulo extensor de puertos}
También se dispone de un módulo extensor de 8 puertos de entrada que permite la conexión de hasta 64 sensores PIR. Para este desarrollo se utiliza el componente PCF8574 de la marca Texas Instruments, el cual dispone de conexión I2C de 100 kHz con microcontrolador y la capacidad de direccionar hasta 8 de los mismos. Su tensión de trabajo es de 5 V y su consumo promedio de 25 mA.
\begin{figure}[H]
	\centering
	\includegraphics[width=5cm, height=5cm]{./Figures/extensorgpio.png}
	\caption{Extensor multipropósito PCF8574.}
	\label{fig:gpio}
\end{figure}

\subsubsection{Módulo de comunicación Bluetooth}
El módulo para la interacción con el usuario utilizado en este trabajo es el HC-05 (en conjunto con la placa de evaluación ZS-040), cuyo protocolo es Bluetooth 2.0 y dispone de una configuración simultanea tanto como maestro y como esclavo. Se conecta al microcontrolador a través de interfaz serie, requiriendo una tensión de 5 V y con un consumo de 100 mA.
\begin{figure}[H]
	\centering
	\includegraphics[width=5cm, height=5cm]{./Figures/BT.jpg}
	\caption{Módulo Bluetooth HC-05 montado sobre ZS-040.}
	\label{fig:bt}
\end{figure}
\subsection{Software}
En esta sección se mencionan los software, lenguajes de programación y herramientas desarrolladas por terceros que fueron utilizados para el desarrollo de este proyecto.

\begin{itemize}
	\item Visual Studio Code: es editor de código fuente gratuito y multiplataforma desarrollado por la empresa Microsoft. Posee un conjunto de extensiones que permite añadir soporte para cualquier lenguaje de programación y depuración.
	\item STM32 CubeIDE: es entorno de desarrollo integrado (IDE, por sus siglas en inglés) de código abierto para programar microcontroladores y microprocesadores de la familia STM32. Permite configurar periféricos, generar código automáticamente, compilarlo y depurarlo en la misma plataforma Se encuentra basado en el entorno de desarrollo Eclipse. 
    \item MIT APP Inventor: es una plataforma gratuita y en línea para la creación de aplicaciones para dispositivos Android. Desarrollado por el Instituto Tecnológico de Massachusetts, permite a los usuarios no técnicos diseñar y crear aplicaciones funcionales a través de un navegador web.
    \item Lenguaje de programación C: es un lenguaje de programación de propósito general e imperativo que permite el control directo del hardware. Fue creado por Dennis Ritchie y Ken Thompson entre 1969 y 1972 y se caracteriza por su eficiencia, portabilidad y la capacidad de manipular memoria mediante punteros, ideal para sistemas operativos, software embebido y controladores de hardware. 
    \item Compilador GCC (GNU Compiler Collection): es un conjunto de compiladores creado por el Proyecto GNU y distribuido como software libre que permite compilar código en lenguaje C, C++, Objective-C, Fortran entre otros. Su propósito es tomar un programa escrito en un lenguaje de alto nivel (como C) y transformarlo en un programa ejecutable en lenguaje de máquina.
    \item Doxigen: es una herramienta de generación automática de documentación técnica a partir de los comentarios en el código fuente de lenguajes como C++, C, Java y Python. Analiza el código para extraer y formatear la información en formatos HTML o PDF y permite mantener la documentación sincronizada con el código.
\end{itemize}

\section{Protocolos de comunicación}

Se recomienda no utilizar \textbf{texto en negritas} en ningún párrafo, ni tampoco texto \underline{subrayado}. En cambio sí se debe utilizar \textit{texto en itálicas} para palabras en un idioma extranjero, al menos la primera vez que aparecen en el texto. En el caso de palabras que estamos inventando se deben utilizar ``comillas'', así como también para citas textuales. Por ejemplo, un \textit{digital filter} es una especie de ``selector'' que permite separar ciertos componentes armónicos en particular.

La escritura debe ser impersonal. Por ejemplo, no utilizar ``el diseño del firmware lo hice de acuerdo con tal principio'', sino ``el firmware fue diseñado utilizando tal principio''. 

El trabajo es algo que al momento de escribir la memoria se supone que ya está concluido, entonces todo lo que se refiera a hacer el trabajo se narra en tiempo pasado, porque es algo que ya ocurrió. Por ejemplo, "se diseñó el firmware empleando la técnica de test driven development".

En cambio, la memoria es algo que está vivo cada vez que el lector la lee. Por eso transcurre siempre en tiempo presente, como por ejemplo:

``En el presente capítulo se da una visión global sobre las distintas pruebas realizadas y los resultados obtenidos. Se explica el modo en que fueron llevados a cabo los test unitarios y las pruebas del sistema''.

Se recomienda no utilizar una sección de glosario sino colocar la descripción de las abreviaturas como parte del mismo cuerpo del texto. Por ejemplo, RTOS (\textit{Real Time Operating System}, Sistema Operativo de Tiempo Real) o en caso de considerarlo apropiado mediante notas a pie de página.

Si se desea indicar alguna página web utilizar el siguiente formato de referencias bibliográficas, dónde las referencias se detallan en la sección de bibliografía de la memoria, utilizado el formato establecido por IEEE en \citep{IEEE:citation}. Por ejemplo, ``el presente trabajo se basa en la plataforma EDU-CIAA-NXP \citep{CIAA}, la cual...''.

\subsection{Figuras} 

Al insertar figuras en la memoria se deben considerar determinadas pautas. Para empezar, usar siempre tipografía claramente legible. Luego, tener claro que \textbf{es incorrecto} escribir por ejemplo esto: ``El diseño elegido es un cuadrado, como se ve en la siguiente figura:''

\begin{figure}[h]
\centering
\includegraphics[scale=.45]{./Figures/cuadradoAzul.png}
\end{figure}

La forma correcta de utilizar una figura es con referencias cruzadas, por ejemplo: ``Se eligió utilizar un cuadrado azul para el logo, como puede observarse en la figura \ref{fig:cuadradoAzul}''.

\begin{figure}[ht]
	\centering
	\includegraphics[scale=.45]{./Figures/cuadradoAzul.png}
	\caption{Ilustración del cuadrado azul que se eligió para el diseño del logo.}
	\label{fig:cuadradoAzul}
\end{figure}

El texto de las figuras debe estar siempre en español, excepto que se decida reproducir una figura original tomada de alguna referencia. En ese caso la referencia de la cual se tomó la figura debe ser indicada en el epígrafe de la figura e incluida como una nota al pie, como se ilustra en la figura \ref{fig:palabraIngles}.

\begin{figure}[htpb]
	\centering
	\includegraphics[scale=.3]{./Figures/word.jpeg}
	\caption{Imagen tomada de la página oficial del procesador\protect\footnotemark.}
	\label{fig:palabraIngles}
\end{figure}

\footnotetext{Imagen tomada de \url{https://goo.gl/images/i7C70w}}

La figura y el epígrafe deben conformar una unidad cuyo significado principal pueda ser comprendido por el lector sin necesidad de leer el cuerpo central de la memoria. Para eso es necesario que el epígrafe sea todo lo detallado que corresponda y si en la figura se utilizan abreviaturas entonces aclarar su significado en el epígrafe o en la misma figura.



\begin{figure}[ht]
	\centering
	\includegraphics[scale=.37]{./Figures/questionMark.png}
	\caption{¿Por qué de pronto aparece esta figura?}
	\label{fig:questionMark}
\end{figure}

Nunca colocar una figura en el documento antes de hacer la primera referencia a ella, como se ilustra con la figura \ref{fig:questionMark}, porque sino el lector no comprenderá por qué de pronto aparece la figura en el documento, lo que distraerá su atención.

Otra posibilidad es utilizar el entorno \textit{subfigure} para incluir más de una figura, como se puede ver en la figura \ref{fig:three graphs}. Notar que se pueden referenciar también las figuras internas individualmente de esta manera: \ref{fig:1de3}, \ref{fig:2de3} y \ref{fig:3de3}.
 
\begin{figure}[!htpb]
     \centering
     \begin{subfigure}[b]{0.3\textwidth}
         \centering
         \includegraphics[width=.65\textwidth]{./Figures/questionMark}
         \caption{Un caption.}
         \label{fig:1de3}
     \end{subfigure}
     \hfill
     \begin{subfigure}[b]{0.3\textwidth}
         \centering
         \includegraphics[width=.65\textwidth]{./Figures/questionMark}
         \caption{Otro.}
         \label{fig:2de3}
     \end{subfigure}
     \hfill
     \begin{subfigure}[b]{0.3\textwidth}
         \centering
         \includegraphics[width=.65\textwidth]{./Figures/questionMark}
         \caption{Y otro más.}
         \label{fig:3de3}
     \end{subfigure}
        \caption{Tres gráficos simples.}
        \label{fig:three graphs}
\end{figure}

El código para generar las imágenes se encuentra disponible para su reutilización en el archivo \file{Chapter2.tex}.

\subsection{Tablas}

Para las tablas utilizar el mismo formato que para las figuras, sólo que el epígrafe se debe colocar arriba de la tabla, como se ilustra en la tabla \ref{tab:peces}. Observar que sólo algunas filas van con líneas visibles y notar el uso de las negritas para los encabezados.  La referencia se logra utilizando el comando \verb|\ref{<label>}| donde label debe estar definida dentro del entorno de la tabla.

\begin{verbatim}
\begin{table}[h]
	\centering
	\caption[caption corto]{caption largo más descriptivo}
	\begin{tabular}{l c c}    
		\toprule
		\textbf{Especie}     & \textbf{Tamaño} & \textbf{Valor}\\
		\midrule
		Amphiprion Ocellaris & 10 cm           & \$ 6.000 \\		
		Hepatus Blue Tang    & 15 cm           & \$ 7.000 \\
		Zebrasoma Xanthurus  & 12 cm           & \$ 6.800 \\
		\bottomrule
		\hline
	\end{tabular}
	\label{tab:peces}
\end{table}
\end{verbatim}


\begin{table}[h]
	\centering
	\caption[caption corto]{caption largo más descriptivo.}
	\begin{tabular}{l c c}    
		\toprule
		\textbf{Especie} 	 & \textbf{Tamaño} 		& \textbf{Valor}  \\
		\midrule
		Amphiprion Ocellaris & 10 cm 				& \$ 6.000 \\		
		Hepatus Blue Tang	 & 15 cm				& \$ 7.000 \\
		Zebrasoma Xanthurus	 & 12 cm				& \$ 6.800 \\
		\bottomrule
		\hline
	\end{tabular}
	\label{tab:peces}
\end{table}

En cada capítulo se debe reiniciar el número de conteo de las figuras y las tablas, por ejemplo, figura 2.1 o tabla 2.1, pero no se debe reiniciar el conteo en cada sección. Por suerte la plantilla se encarga de esto por nosotros.

\subsection{Ecuaciones}
\label{sec:Ecuaciones}

Al insertar ecuaciones en la memoria dentro de un entorno \textit{equation}, éstas se numeran en forma automática  y se pueden referir al igual que como se hace con las figuras y tablas, por ejemplo ver la ecuación \ref{eq:metric}.

\begin{equation}
	\label{eq:metric}
	ds^2 = c^2 dt^2 \left( \frac{d\sigma^2}{1-k\sigma^2} + \sigma^2\left[ d\theta^2 + \sin^2\theta d\phi^2 \right] \right)
\end{equation}
                                                        
Para generar la ecuación \ref{eq:metric} se utilizó el siguiente código:

\begin{verbatim}
\begin{equation}
	\label{eq:metric}
	ds^2 = c^2 dt^2 \left( \frac{d\sigma^2}{1-k\sigma^2} + 
	\sigma^2\left[ d\theta^2 + 
	\sin^2\theta d\phi^2 \right] \right)
\end{equation}
\end{verbatim}
