%----------------------------------------------------------------------------------------
% Preambulo y Configuración
%----------------------------------------------------------------------------------------

\documentclass[
    11pt,
    spanish,
    singlespacing,
    parskip,
    headsepline,
    bookmarks=true,
    unicode=true,
    pdftoolbar=true,
    pdfmenubar=true,
    pdffitwindow=false,
    colorlinks=true,
    linkcolor=blue,
    citecolor=blue,
    urlcolor=blue
]{MastersDoctoralThesis}

\usepackage[utf8]{inputenc} % Codificación de entrada UTF-8
\usepackage[T1]{fontenc}    % Codificación de salida para caracteres especiales
\usepackage{graphicx}       % Manejo de gráficos
\usepackage{eso-pic}        % Permite agregar fondos
\usepackage{hyperref}       % Manejo de hipervínculos y marcadores

% Redefinición de caracteres problemáticos en marcadores
\hypersetup{
    pdftitle={Sistema inteligente de iluminación de escaleras para el hogar},
    pdfauthor={Ing. Pedro Santiago Escribá},
    pdfkeywords={Sistemas Embebidos, Internet de las Cosas, Inteligencia Artificial},
    pdfstartview={FitH},
    unicode=true,
    colorlinks=true,
    linkcolor=blue,
    citecolor=blue,
    urlcolor=blue
}

\pdfstringdefDisableCommands{%
  \def\texttt#1{#1}%
  \def\textbf#1{#1}%
  \def\textit#1{#1}%
  \def\"{\"}%
  \def\~{~}%
  \def\'{'}%
  \def\^{}%
  \def\textunderscore{\_} % Manejo del subrayado en marcadores
}


% Definir comandos requeridos por la clase
\newcommand{\degreename}{Maestría en Ciencias} % Cambia según tu título
\newcommand{\univname}{Universidad Nacional de Ejemplo} % Cambia según tu universidad
\newcommand{\keywordnames}{Palabras clave:}
%----------------------------------------------------------------------------------------
% Documento Principal
%----------------------------------------------------------------------------------------

\begin{document}

% Configuración de la portada
\posgrado{Carrera / Maestría}
\keywords{Sistemas Embebidos, Internet de las Cosas, Inteligencia Artificial}

% Incluir la portada desde un archivo separado
\include{portada}

% Configuración del contenido preliminar
\frontmatter % Usar numeración romana para las páginas preliminares
\pagestyle{plain} % Estilo de encabezado simple

%----------------------------------------------------------------------------------------
% Resumen
%----------------------------------------------------------------------------------------

\begin{abstract}
\addchaptertocentry{\abstractname} % Agregar resumen al índice
La presente monografía describe el trabajo realizado para el desarrollo de un sistema de iluminación de escaleras para el hogar. Detalla sus generalidades, los pasos efectuados para la ejecución del software y las pruebas de calidad del mismo. También busca que el lector pueda interiorizarse en los conceptos aplicados así como en el funcionamiento general de lo desarrollado.

Por otro lado, es objetivo de esta memoria dejar por escrito los resultados obtenidos a partir del trabajo efectuado, con la intención de que estos sean de utilidad para quienes decidan continuar con su desarrollo.				
						
\end{abstract}

%----------------------------------------------------------------------------------------
% Agradecimientos
%----------------------------------------------------------------------------------------

\begin{acknowledgements}
\vspace{1.5cm}
A mi familia, amigos y compañeros.
\end{acknowledgements}

%----------------------------------------------------------------------------------------
% Índice
%----------------------------------------------------------------------------------------

\tableofcontents
\listoffigures
\listoftables

%----------------------------------------------------------------------------------------
% Capítulos
%----------------------------------------------------------------------------------------

\mainmatter % Iniciar numeración numérica para el contenido principal
\pagestyle{thesis} % Estilo de encabezado de tesis

% Incluir capítulos desde archivos separados
% Chapter 1

\chapter{Introducción general} % Main chapter title

\label{Chapter1} % For referencing the chapter elsewhere, use \ref{Chapter1} 
\label{IntroGeneral}

En este capítulo se introducen conceptos básicos de los sistemas de iluminación modernos, también se presentan los objetivos, la motivación, los alcances de este trabajo y se analiza el estado del arte.

%----------------------------------------------------------------------------------------

% Define some commands to keep the formatting separated from the content 
\newcommand{\keyword}[1]{\textbf{#1}}
\newcommand{\tabhead}[1]{\textbf{#1}}
\newcommand{\code}[1]{\texttt{#1}}
\newcommand{\file}[1]{\texttt{\bfseries#1}}
\newcommand{\option}[1]{\texttt{\itshape#1}}
\newcommand{\grados}{$^{\circ}$}

%----------------------------------------------------------------------------------------

%\section{Introducción}

%----------------------------------------------------------------------------------------
\section{Introducción a los sistemas de iluminación}

A partir de la llegada de los LEDs ({\it{Light Emissor Doide}}) de alta potencia al mercado de la electrónica, los sistemas de iluminación del hogar (y en general) experimentaron una verdadera revolución comercial y tecnológica. Los bajos costos y la posibilidad de control generaron un sin fin innovador que sigue vigente hoy en día. En una casa tipo es común observar lámparas LED de colores brillantes capaces de ser controladas con un teléfono celular y producir un efecto visual poco pensado algunas décadas en el pasado o, también, su capacidad de ser encendidas o apagadas con un simple ''click programado''. Sin embargo, aún constituye un desafío para los ingenieros integrar estos sistemas de iluminación a la arquitectura edilicia manteniendo eficiencia, estética y bajo costo.

\subsection{Iluminación en escaleras modernas}

Uno de los mayores problemas de la iluminación de los hogares modernos es lograr un efecto lumínico vistoso sobre espacios asimétricos, con un número elevado de bordes irregulares donde las sombras juegan un desafío. Un claro ejemplo son las escaleras, donde los propios escalones proyectan sombras sobre sus predecesores o antecesores. 

La ingeniería en iluminación se encarga de determinar la cantidad de luz que requiere un espacio, en función de sus características y de las normas vigentes. El Código Residencial Internacional \citep{CODRED} recomienda 100 lúmenes por pie cuadrado (alrededor de 100 centímetros cuadrados) en sectores de interior con escaleras y cuya temperatura de color ronde entre 2700 y 3000 kelvin.

Sin embargo, hay otro aspecto que este tipo de códigos no contemplan del todo: la eficiencia energética. Este factor es fundamental a la hora de efectuar una obra lumínica en un hogar, ya que define la estrategia constructiva de la escalera.

\subsection{Eficiencia energética}

Tanto la disponibilidad de energía eléctrica como la potencia requerida por el sistema de iluminación constituyen aspectos clave en el diseño de una instalación lumínica para una escalera. Si se dispone de pocos puntos de alimentación eléctrica, la oportunidad de agregar puntos de luz al espacio estará completamente limitado, lo que afecta la capacidad de cubrirlo con los niveles de lúmenes adecuados. Además, deben considerarse el consumo total sobre dicho acceso energético y la capacidad de la infraestructura para soportarlo.

A pesar de estos inconvenientes, los avances tecnológicos han generado soluciones a partir de la invención de las tiras LED, cuyo consumo energético es alrededor de un 90\% menor que el de un sistema de iluminación tradicional. De manera general, estas tiras logran iluminar 100 lúmenes por vatio, mientras que una bombilla incandescente ofrece aproximadamente 20 lúmenes por vatio. Además, su extensión lineal permite distribuir la energía eléctrica desde un único punto de alimentación a lo largo de toda la escalera sin requerir cableado adicional.

%----------------------------------------------------------------------------------------

\section{Estado del arte}

En esta sección se presentan algunos productos similares al desarrollado en este documento que se encuentran disponibles en el mercado. Se realiza un breve análisis de las prestaciones y costos, y se describen las diferencias y similitudes con la solución propuesta en este trabajo.
 
\subsection{SuperlightingLED}

La empresa de tecnología LED SuperlightingLED \citep{SuperlightingLED} ofrece en el mercado internacional un sistema de iluminación de escaleras con tecnología de tiras LEDs y sensores de movimientos (SCSSLKIT-12V-COB). 

Según la información publicada en su página de ventas, el sistema incluye un controlador ES32, que actúa como unidad de control principal, y dos sensores de movimiento ubicados en los extremos de la escalera. Es importante aclarar que el ES32 no es un microcontrolador, sino un controlador dedicado para tiras LED, diseñado para gestionar efectos de iluminación, la activación por sensores y la intensidad del brillo. Además, cuenta con una fuente de alimentación de 12 V y las tiras LED (de color y temperatura variables) para cubrir hasta 32 escalones de un ancho total de 1,2 m cada uno y que deben cablearse hasta la unidad controladora. 

El sistema se comercializa en dos versiones principales: una para interior (IP20) y otra para exterior (IP67). Adicionalmente, puede incluir perfiles de aluminio para facilitar su instalación. 

En las figuras \ref{fig:spled1} y \ref{fig:spled2} se pueden apreciar los componentes de este sistema que actualmente tiene un costo de unos USD 160 en su versión más económica y hasta USD 630 en su versión premium.

\begin{figure}[H]
	\centering
	\includegraphics[width=7cm, height=7cm]{./Figures/superlightingLED.png}
	\caption{Sistema de SuperlightingLED y su conexión.\protect\footnotemark}

	\label{fig:spled1}
\end{figure}
\footnotetext{\url{https://www.superlightingled.com/}}


\begin{figure}[H]
    \centering
    \includegraphics[width=7cm, height=7cm]{./Figures/superlightingLED2.png}
    \caption{Componentes incluidos en el paquete básico.\protect\footnotemark}
    \label{fig:spled2}
\end{figure}
\footnotetext{\url{https://www.superlightingled.com/}}

\subsection{GIDEALED}

Otro producto presente en el mercado que se lo puede vincular con el trabajo desarrollado en esta memoria, es el sistema de iluminación de escaleras de la empresa GIDEALED \citep{GIDEALED} modelo RL-STEP-09. Este sistema es apto para hasta 20 escalones de un metro de ancho, funciona a 12 V y consta de dos sensores de movimientos del tipo PIR (\textit{Passive Infrared Sensor}, por sus siglas en inglés) a instalar en los extremos de la escalera, además de un control remoto infrarrojo para el encendido y apagado. Las tiras LED incluidas en el kit son de color fijo, de temperatura de color de 3000 K, e incorporan control de brillo e intensidad. No cuentan con protección frente al agua o humedad. La figura \ref{fig:gidealed} muestra los componentes del kit, que se encuentra a la venta en la República Argentina por un precio aproximado de \$210000.

\begin{figure}[H]
	\centering
	\includegraphics[width=7cm, height=7cm]{./Figures/GIDEALED.png}
	\caption{Sistema para escaleras de GIDEALED.\protect\footnotemark}
	\label{fig:gidealed}
\end{figure}
\footnotetext{\url{https://www.amazon.com/stores/GIDEALED/page/55B6DBC4-21FB-439B-AC80-13FDE4121084}}

\subsection{Análisis del estado del arte y aportes del desarrollo}

Luego de realizar un análisis exhaustivo de los distintos productos disponibles en el mercado, se concluye que, si bien existen soluciones que abordan la problemática planteada, la mayoría de ellas carece del grado de innovación y personalización que se busca en este desarrollo. La incorporación de sensores individuales en cada escalón permite un control de iluminación más preciso y adaptativo, constituyendo uno de los principales factores diferenciales del sistema propuesto.
A ello se suman su bajo costo de implementación, la posibilidad de instalación directa por parte del usuario sin asistencia técnica y la integración de tecnologías modernas de control.

En la siguiente lista se destacan las principales ventajas del sistema frente a los productos analizados:
\begin{itemize}
    \item Bajo costo de desarrollo y mantenimiento.
    \item Instalación sencilla, sin necesidad de herramientas especializadas.
    \item Mayor precisión y personalización del control lumínico.
    \item Posibilidad de control mediante teléfono celular vía conexión Bluetooth.
    \item Configuración en conexión serie de los módulos de luz.
    \item Compatibilidad con escaleras de hasta 64 escalones.
\end{itemize}
%----------------------------------------------------------------------------------------

\section{Motivación}

Si bien existen múltiples sistemas de iluminación destinados al hogar, las escaleras resultan un ámbito poco explorado en cuanto a propuestas innovadoras de diseño lumínico. Este espacio, por sus características arquitectónicas y su uso cotidiano, ofrece un amplio potencial para la experimentación con nuevas formas de iluminación funcional y estética. En este sentido, los avances en electrónica y domótica desempeñan un papel fundamental al permitir el desarrollo de sistemas más inteligentes, eficientes y adaptables. Gracias a ellos, el usuario no solo puede decidir qué iluminar, sino también cómo hacerlo, ajustando la intensidad, el color o el comportamiento de la luz según sus preferencias o necesidades específicas.

Por otra parte, la disponibilidad de sensores, actuadores y microcontroladores de bajo costo ha permitido el acceso a tecnologías que antes eran exclusivas de instalaciones complejas o costosas. A esto se suman los sistemas de comunicación inalámbrica, que facilitan la implementación sin requerir cableados extensos ni modificaciones estructurales. Estos factores en conjunto constituyen una motivación central para generar un nuevo módulo lumínico transformable que responda a las necesidades de un usuario particular. 


%----------------------------------------------------------------------------------------

\section{Objetivos y alcances}

Este trabajo de desarrollo pretende integrar los avances e innovaciones tecnológicas en la iluminación del hogar agregando valor a bajo costo y control por parte del usuario. Es así que, integrando los conocimientos adquiridos a lo largo de la Carrera de Especialización en Sistemas Embebidos de la FIUBA, se desarrolla el firmware en lenguaje de programación C de un sistema embebido basado en tiras LED y controlado mediante sensores de proximidad que permita iluminar una escalera de una cantidad fija de escalones. También es parte de este trabajo la incorporación de un control mediante tecnología Bluetooth de la tonalidad de la luz, brillo y color, así como su encendido o apagado de acuerdo a la circulación del usuario por la escalera.

Por otro lado, este trabajo no contempla el desarrollo de un hardware integral, sino la elaboración de un prototipo funcional basado en componentes provistos por fabricantes comerciales. Tampoco se contempla el desarrollo de una interfaz de usuario ni soporte postventa.
%----------------------------------------------------------------------------------------
\chapter{Introducción específica} % Main chapter title

\label{Chapter2}

%----------------------------------------------------------------------------------------
%	SECTION 1
%----------------------------------------------------------------------------------------
En este capitulo se mencionan los componentes, software y programas de terceros utilizados para el desarrollo del sistema de esta memoria.

\section{Componentes del sistema}
\label{sec:ejemplo}

El sistema de iluminación cuenta con un conjunto de componentes necesarios para su funcionamiento, que van desde lenguajes de programación hasta pequeños componentes electrónicos. A continuación se hace mención a cada uno de ellos con una breve descripción. 

\subsection{Hardware}

\subsubsection{Microcontrolador}
Este proyecto utiliza un microcontrolador de 32 bits de la marca STM32 modelo F4 basado en el núcleo Arm® Cortex®-M4 con un frecuencia de hasta 180 MHz. Incluye una unidad de punto flotante de precisión simple e incorporan memorias embebidas de alta velocidad (memoria Flash de 2 Mbytes y hasta 256 Kbytes de SRAM). Cuenta con un gran numero de entradas/salidas, convertidores analógicos digitales, periféricos de modulación de pulsos y hasta 12 temporizadores. Posee además de un gran numero de interfaces de comunicación entre las que se destacan UART, I2C, SMBus y SPI. Para el desarrollo de este proyecto se utiliza una placa de evaluación Núcleo de la marca STM modelo F429ZI la cual dispone de todos los elementos necesarios para el uso del microcontrolador.

\subsubsection{Tira LED pixel}
Para el desarrollo del sistema de iluminación se dispone de una tira LED del tipo pixel. La misma consta de tres diodos emisores de luz (uno para cada color primario) en conjunto con un controlador modelo WS2812B. Este controlador incluye un registro de datos digitales, un circuito de amplificación y reformado de señal, en conjunto con un oscilador interno de alta precisión. Utiliza un protocolo de comunicación NZR de un solo cable donde el puerto de entrada recibe los datos enviados por el microcontrolador son almacenados en un registro interno y transferidos por el puerto de salida al siguiente dispositivo conectado en cadena. El dato consta de 24 pulsos (8 por color), donde un cero lógico debe estar compuesto por 0.4 us en alto y 0.85 en bajo y un uno lógico por 0.8us en alto y 0.45 us en bajo.

El voltaje de trabajo es de 5 V y el consumo depende de la cantidad de LEDs de la tira, con un consumo cercano a 1 uA por LED.

\subsubsection{Modulo extensor de puertos}
También se dispone de un modulo extensor de 8 puertos de entrada que permite la conexión de hasta 64 sensores en la escalera. Para este desarrollo se utiliza el modulo I2C PCF8574 de la marca Texas Instruments

\begin{itemize}
	\item Este es el primer elemento de la lista.
	\item Este es el segundo elemento de la lista.
\end{itemize}

Notar el uso de las mayúsculas y el punto al final de cada elemento.

Si se desea poner una lista numerada el formato es este:

\begin{enumerate}
	\item Este es el primer elemento de la lista.
	\item Este es el segundo elemento de la lista.
\end{enumerate}

Notar el uso de las mayúsculas y el punto al final de cada elemento.

\subsection{Este es el título de una subsección}
\label{subsec:ejemplo}

Se recomienda no utilizar \textbf{texto en negritas} en ningún párrafo, ni tampoco texto \underline{subrayado}. En cambio sí se debe utilizar \textit{texto en itálicas} para palabras en un idioma extranjero, al menos la primera vez que aparecen en el texto. En el caso de palabras que estamos inventando se deben utilizar ``comillas'', así como también para citas textuales. Por ejemplo, un \textit{digital filter} es una especie de ``selector'' que permite separar ciertos componentes armónicos en particular.

La escritura debe ser impersonal. Por ejemplo, no utilizar ``el diseño del firmware lo hice de acuerdo con tal principio'', sino ``el firmware fue diseñado utilizando tal principio''. 

El trabajo es algo que al momento de escribir la memoria se supone que ya está concluido, entonces todo lo que se refiera a hacer el trabajo se narra en tiempo pasado, porque es algo que ya ocurrió. Por ejemplo, "se diseñó el firmware empleando la técnica de test driven development".

En cambio, la memoria es algo que está vivo cada vez que el lector la lee. Por eso transcurre siempre en tiempo presente, como por ejemplo:

``En el presente capítulo se da una visión global sobre las distintas pruebas realizadas y los resultados obtenidos. Se explica el modo en que fueron llevados a cabo los test unitarios y las pruebas del sistema''.

Se recomienda no utilizar una sección de glosario sino colocar la descripción de las abreviaturas como parte del mismo cuerpo del texto. Por ejemplo, RTOS (\textit{Real Time Operating System}, Sistema Operativo de Tiempo Real) o en caso de considerarlo apropiado mediante notas a pie de página.

Si se desea indicar alguna página web utilizar el siguiente formato de referencias bibliográficas, dónde las referencias se detallan en la sección de bibliografía de la memoria, utilizado el formato establecido por IEEE en \citep{IEEE:citation}. Por ejemplo, ``el presente trabajo se basa en la plataforma EDU-CIAA-NXP \citep{CIAA}, la cual...''.

\subsection{Figuras} 

Al insertar figuras en la memoria se deben considerar determinadas pautas. Para empezar, usar siempre tipografía claramente legible. Luego, tener claro que \textbf{es incorrecto} escribir por ejemplo esto: ``El diseño elegido es un cuadrado, como se ve en la siguiente figura:''

\begin{figure}[h]
\centering
\includegraphics[scale=.45]{./Figures/cuadradoAzul.png}
\end{figure}

La forma correcta de utilizar una figura es con referencias cruzadas, por ejemplo: ``Se eligió utilizar un cuadrado azul para el logo, como puede observarse en la figura \ref{fig:cuadradoAzul}''.

\begin{figure}[ht]
	\centering
	\includegraphics[scale=.45]{./Figures/cuadradoAzul.png}
	\caption{Ilustración del cuadrado azul que se eligió para el diseño del logo.}
	\label{fig:cuadradoAzul}
\end{figure}

El texto de las figuras debe estar siempre en español, excepto que se decida reproducir una figura original tomada de alguna referencia. En ese caso la referencia de la cual se tomó la figura debe ser indicada en el epígrafe de la figura e incluida como una nota al pie, como se ilustra en la figura \ref{fig:palabraIngles}.

\begin{figure}[htpb]
	\centering
	\includegraphics[scale=.3]{./Figures/word.jpeg}
	\caption{Imagen tomada de la página oficial del procesador\protect\footnotemark.}
	\label{fig:palabraIngles}
\end{figure}

\footnotetext{Imagen tomada de \url{https://goo.gl/images/i7C70w}}

La figura y el epígrafe deben conformar una unidad cuyo significado principal pueda ser comprendido por el lector sin necesidad de leer el cuerpo central de la memoria. Para eso es necesario que el epígrafe sea todo lo detallado que corresponda y si en la figura se utilizan abreviaturas entonces aclarar su significado en el epígrafe o en la misma figura.



\begin{figure}[ht]
	\centering
	\includegraphics[scale=.37]{./Figures/questionMark.png}
	\caption{¿Por qué de pronto aparece esta figura?}
	\label{fig:questionMark}
\end{figure}

Nunca colocar una figura en el documento antes de hacer la primera referencia a ella, como se ilustra con la figura \ref{fig:questionMark}, porque sino el lector no comprenderá por qué de pronto aparece la figura en el documento, lo que distraerá su atención.

Otra posibilidad es utilizar el entorno \textit{subfigure} para incluir más de una figura, como se puede ver en la figura \ref{fig:three graphs}. Notar que se pueden referenciar también las figuras internas individualmente de esta manera: \ref{fig:1de3}, \ref{fig:2de3} y \ref{fig:3de3}.
 
\begin{figure}[!htpb]
     \centering
     \begin{subfigure}[b]{0.3\textwidth}
         \centering
         \includegraphics[width=.65\textwidth]{./Figures/questionMark}
         \caption{Un caption.}
         \label{fig:1de3}
     \end{subfigure}
     \hfill
     \begin{subfigure}[b]{0.3\textwidth}
         \centering
         \includegraphics[width=.65\textwidth]{./Figures/questionMark}
         \caption{Otro.}
         \label{fig:2de3}
     \end{subfigure}
     \hfill
     \begin{subfigure}[b]{0.3\textwidth}
         \centering
         \includegraphics[width=.65\textwidth]{./Figures/questionMark}
         \caption{Y otro más.}
         \label{fig:3de3}
     \end{subfigure}
        \caption{Tres gráficos simples.}
        \label{fig:three graphs}
\end{figure}

El código para generar las imágenes se encuentra disponible para su reutilización en el archivo \file{Chapter2.tex}.

\subsection{Tablas}

Para las tablas utilizar el mismo formato que para las figuras, sólo que el epígrafe se debe colocar arriba de la tabla, como se ilustra en la tabla \ref{tab:peces}. Observar que sólo algunas filas van con líneas visibles y notar el uso de las negritas para los encabezados.  La referencia se logra utilizando el comando \verb|\ref{<label>}| donde label debe estar definida dentro del entorno de la tabla.

\begin{verbatim}
\begin{table}[h]
	\centering
	\caption[caption corto]{caption largo más descriptivo}
	\begin{tabular}{l c c}    
		\toprule
		\textbf{Especie}     & \textbf{Tamaño} & \textbf{Valor}\\
		\midrule
		Amphiprion Ocellaris & 10 cm           & \$ 6.000 \\		
		Hepatus Blue Tang    & 15 cm           & \$ 7.000 \\
		Zebrasoma Xanthurus  & 12 cm           & \$ 6.800 \\
		\bottomrule
		\hline
	\end{tabular}
	\label{tab:peces}
\end{table}
\end{verbatim}


\begin{table}[h]
	\centering
	\caption[caption corto]{caption largo más descriptivo.}
	\begin{tabular}{l c c}    
		\toprule
		\textbf{Especie} 	 & \textbf{Tamaño} 		& \textbf{Valor}  \\
		\midrule
		Amphiprion Ocellaris & 10 cm 				& \$ 6.000 \\		
		Hepatus Blue Tang	 & 15 cm				& \$ 7.000 \\
		Zebrasoma Xanthurus	 & 12 cm				& \$ 6.800 \\
		\bottomrule
		\hline
	\end{tabular}
	\label{tab:peces}
\end{table}

En cada capítulo se debe reiniciar el número de conteo de las figuras y las tablas, por ejemplo, figura 2.1 o tabla 2.1, pero no se debe reiniciar el conteo en cada sección. Por suerte la plantilla se encarga de esto por nosotros.

\subsection{Ecuaciones}
\label{sec:Ecuaciones}

Al insertar ecuaciones en la memoria dentro de un entorno \textit{equation}, éstas se numeran en forma automática  y se pueden referir al igual que como se hace con las figuras y tablas, por ejemplo ver la ecuación \ref{eq:metric}.

\begin{equation}
	\label{eq:metric}
	ds^2 = c^2 dt^2 \left( \frac{d\sigma^2}{1-k\sigma^2} + \sigma^2\left[ d\theta^2 + \sin^2\theta d\phi^2 \right] \right)
\end{equation}
                                                        
Para generar la ecuación \ref{eq:metric} se utilizó el siguiente código:

\begin{verbatim}
\begin{equation}
	\label{eq:metric}
	ds^2 = c^2 dt^2 \left( \frac{d\sigma^2}{1-k\sigma^2} + 
	\sigma^2\left[ d\theta^2 + 
	\sin^2\theta d\phi^2 \right] \right)
\end{equation}
\end{verbatim}

\chapter{Diseño e implementación} % Main chapter title

\label{Chapter3} % Change X to a consecutive number; for referencing this chapter elsewhere, use \ref{ChapterX}

En este capítulo se expone el diseño del producto y la implementación de su arquitectura. Se describen los elementos que lo componen y la organización interna del sistema. Para ello se utilizan diagramas de bloques y de flujo de información, que permiten visualizar la estructura general y comprender cómo se articula cada componente dentro del conjunto.

\definecolor{mygreen}{rgb}{0,0.6,0}
\definecolor{mygray}{rgb}{0.5,0.5,0.5}
\definecolor{mymauve}{rgb}{0.58,0,0.82}

%%%%%%%%%%%%%%%%%%%%%%%%%%%%%%%%%%%%%%%%%%%%%%%%%%%%%%%%%%%%%%%%%%%%%%%%%%%%%
% parámetros para configurar el formato del código en los entornos lstlisting
%%%%%%%%%%%%%%%%%%%%%%%%%%%%%%%%%%%%%%%%%%%%%%%%%%%%%%%%%%%%%%%%%%%%%%%%%%%%%
\lstset{ %
  backgroundcolor=\color{white},   % choose the background color; you must add \usepackage{color} or \usepackage{xcolor}
  basicstyle=\footnotesize,        % the size of the fonts that are used for the code
  breakatwhitespace=false,         % sets if automatic breaks should only happen at whitespace
  breaklines=true,                 % sets automatic line breaking
  captionpos=b,                    % sets the caption-position to bottom
  commentstyle=\color{mygreen},    % comment style
  deletekeywords={...},            % if you want to delete keywords from the given language
  %escapeinside={\%*}{*)},          % if you want to add LaTeX within your code
  %extendedchars=true,              % lets you use non-ASCII characters; for 8-bits encodings only, does not work with UTF-8
  %frame=single,	                % adds a frame around the code
  keepspaces=true,                 % keeps spaces in text, useful for keeping indentation of code (possibly needs columns=flexible)
  keywordstyle=\color{blue},       % keyword style
  language=[ANSI]C,                % the language of the code
  %otherkeywords={*,...},           % if you want to add more keywords to the set
  numbers=left,                    % where to put the line-numbers; possible values are (none, left, right)
  numbersep=5pt,                   % how far the line-numbers are from the code
  numberstyle=\tiny\color{mygray}, % the style that is used for the line-numbers
  rulecolor=\color{black},         % if not set, the frame-color may be changed on line-breaks within not-black text (e.g. comments (green here))
  showspaces=false,                % show spaces everywhere adding particular underscores; it overrides 'showstringspaces'
  showstringspaces=false,          % underline spaces within strings only
  showtabs=false,                  % show tabs within strings adding particular underscores
  stepnumber=1,                    % the step between two line-numbers. If it's 1, each line will be numbered
  stringstyle=\color{mymauve},     % string literal style
  tabsize=2,	                   % sets default tabsize to 2 spaces
  title=\lstname,                  % show the filename of files included with \lstinputlisting; also try caption instead of title
  morecomment=[s]{/*}{*/}
}


%----------------------------------------------------------------------------------------
%	SECTION 1
%----------------------------------------------------------------------------------------
\section{Diseño del sistema}
Esta sección analiza el diseño del sistema de iluminación, la conectividad entre los diferentes componentes y la interacción entre ellos. El diagrama de la figura \ref{fig:bloques} presenta, en forma de bloques, la composición del producto diferenciada en un conjunto de módulos primarios: control, comunicación con el usuario y de ingreso y egreso de señales. También permite visualizar cómo interactúa cada bloque con los demás y el flujo de información entre ellos.

\begin{figure}[h]
	\centering
	\includegraphics[width=10cm, height=5cm]{./Figures/dbloques.jpg}
	\caption{Diagrama en bloques del sistema.}
	\label{fig:bloques}
\end{figure}

El bloque central del sistema es el de control. Está compuesto por el microcontrolador y se encarga del procesamiento de las señales de ingreso, de la generación de las señales de salida y de la comunicación con los diferentes sensores mediante los protocolos mencionados en la sección \ref{sec:protocolos}. También admnistra el almacenamiento, en la memoria interna, de toda la información vinculada a la configuración de hardware del producto.

Por otro lado, el bloque de comunicación con el usuario está compuesto por el módulo Bluetooth, que permite una interacción bidireccional. De este modo, el usuario puede enviar comandos específicos de configuración del hardware y, además, realizar acciones de control como encendido, ajuste de brillo o selección de color. Este bloque se conecta al microcontrolador mediante protocolo UART.

Finalmente, el bloque de señales se divide en dos submódulos: uno de ingreso de información al sistema y otro de respuesta. El ingreso está conformado por señales binarias provenientes de los sensores digitales de movimiento PIR, que detectan el desplazamiento escalón por escalón. Si bien estos sensores podrían conectarse directamente al microcontrolador mediante sus puertos específicos, se optó por utilizar extensores de puertos que se comunican con el módulo de control a través del protocolo I\textsuperscript{2}C. Una vez captados y procesados los datos, se genera la señal de egreso, necesaria por los LEDs pixel de la tira. Para ello, el módulo PWM del microcontrolador produce una serie de pulsos binarios con diferentes ciclos de trabajo y una frecuencia de 800 kHz, en cumplimiento de las especificaciones técnicas del controlador de los LEDs.

\section{Análisis del potencia}
Si bien el sistema de iluminación desarrollado en este trabajo constituye un prototipo funcional, este debe cumplir con los requisitos mínimos de tensión y corriente para operar dentro de los parámetros esperados. En primer lugar, cabe destacar que el hardware fue diseñado para iluminar cuatro escalones de un metro de ancho, utilizando de los componentes listados a continuación y ya descriptos en el capítulo \ref{Chapter2}.
\begin{itemize}
    \item Una placa de evaluación STM32 Nucleo F429ZI.
    \item Cuatro metros de tira LED pixel WS2812B.
    \item Cuatro sensores de movimiento PIR HC-SR501.
    \item Un módulo Bluetooth HC-05 sobre placa de evaluación ZS-040.
    \item Un módulo extensor de puertos PCF8574.
    \item Una protoboard, cables, transistores y resistencias.
\end{itemize}

De acuerdo a las diferentes hojas de datos todos los componentes mencionados trabajan a 5 V, por lo que puede utilizarse la misma fuente de energía para todos ellos. Sin embargo, la potencia difiere significativamente, siendo la tira LED el elemento de mayor demanda energética. Cada LED de la tira consume un máximo aproximado de 60 mA y, considerando que cada metro contiene 60 LED, la potencia requerida asciende a 3,6 A (o 18 W) por escalón.

En cuanto al microcontrolador, su hoja de datos indica un consumo de 260 µA/MHz, lo que para una operación a 180 MHz implica un consumo cercano a 240 mA. Los sensores PIR requieren 300 mW cada uno, mientras que los extensores de puertos demandan alrededor de 500 mW. El módulo Bluetooth, considerando su placa de evaluación, especifica un consumo máximo de 200 mW. 

Finalmente, el prototipo para cuatro escalones consume aproximadamente 74 W. Cabe destacar que el 97\% del consumo total recae en la iluminación de la escalera, por lo que se dispone de una única fuente de energía que responda a las necesidades expuestas anteriormente, para todos los elementos del sistema.

\section{Análisis del software}
 
La idea de esta sección es resaltar los problemas encontrados, los criterios utilizados y la justificación de las decisiones que se hayan tomado.

Se puede agregar código o pseudocódigo dentro de un entorno lstlisting con el siguiente código:

\begin{verbatim}
\begin{lstlisting}[caption= "un epígrafe descriptivo"]
	las líneas de código irían aquí...
\end{lstlisting}
\end{verbatim}

A modo de ejemplo, se muestra el fragmento de código \ref{cod:vControl}:

\begin{lstlisting}[label=cod:vControl,caption=Pseudocódigo del lazo principal de control.]  % Start your code-block

#define MAX_SENSOR_NUMBER 3
#define MAX_ALARM_NUMBER  6
#define MAX_ACTUATOR_NUMBER 6

uint32_t sensorValue[MAX_SENSOR_NUMBER];		
FunctionalState alarmControl[MAX_ALARM_NUMBER];	//ENABLE or DISABLE
state_t alarmState[MAX_ALARM_NUMBER];						//ON or OFF
state_t actuatorState[MAX_ACTUATOR_NUMBER];			//ON or OFF

void vControl() {

	initGlobalVariables();
	
	period = 500 ms;
		
	while(1) {

		ticks = xTaskGetTickCount();
		
		updateSensors();
		
		updateAlarms();
		
		controlActuators();
		
		vTaskDelayUntil(&ticks, period);
	}
}
\end{lstlisting}




\include{Chapters/Chapter4}
\include{Chapters/Chapter5}

%----------------------------------------------------------------------------------------
% Apéndices
%----------------------------------------------------------------------------------------

\appendix

% Incluir apéndices desde archivos separados si es necesario
%\include{Appendices/AppendixA}

%----------------------------------------------------------------------------------------
% Bibliografía
%----------------------------------------------------------------------------------------

\renewcommand{\bibname}{Bibliografía} % Para asegurarte de que el título sea correcto
\phantomsection % Necesario para que el enlace del marcador sea correcto

\printbibliography[heading=bibintoc]

\end{document}






